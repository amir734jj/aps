\documentclass[12pt]{article}
  
\usepackage{fullpage}
\usepackage{pict2e}

\usepackage{alltt}
\usepackage{amssymb}
\usepackage{amsthm}

\def\implies{\Rightarrow}

\theoremstyle{definition}
\newtheorem{definition}{Definition}[section]

\theoremstyle{theorem}
\newtheorem{consequence}{Consequence}[section]

\def\aset#1{\left\{{#1}\right\}}

\unitlength 1mm

\begin{document}

\title{Quasi Ordering for Dependency Scheduling in the Presence of Cycles}
\author{John Boyland and Amir Hesamian}
\date{DRAFT \today}

\maketitle

\begin{abstract}
  Unlike many graphs, a dependency graph may have loops (edges from a
  vertex to itself).  The presence of a loop makes a difference to the
  graph, since it represents a self-dependency.  Thus representing
  (the transitive closure of) a dependency graph as a partial order
  over strongly connected components loses some information: are the
  single-element components involved in a self-dependency or not?
  This information is needed for an evaluation (scheduling)
  mechanism.  We propose a new algebraic structure: the \emph{quasi
  order}, for which we propose partial and total versions.  Any
  transitive dependency relation can be represented as a quasi partial
  order, and any schedule of such a relation can used a quasi total order. 
\end{abstract}

\section{Motivation}

Consider the following set of equations that should be mutually
satisfied, and its dependency graph shown on the right:
\begin{quote}
  \begin{minipage}{2.75in}
\begin{verbatim}
A = m()
B = n(A)
C = p(A,F)
D = q(A,B,C)
E = r(B,D,E)
F = s(D)
G = t(D,F)
\end{verbatim}
  \end{minipage}\hfill
  \begin{minipage}{2.75in}
    \begin{picture}(40,44)(0,-4)
      \put(10,40){\makebox(0,0){A}}
      \put(30,40){\makebox(0,0){B}}
      \put(0,20){\makebox(0,0){C}}
      \put(20,20){\makebox(0,0){D}}
      \put(40,20){\makebox(0,0){E}}
      \put(10,0){\makebox(0,0){F}}
      \put(30,0){\makebox(0,0){G}}
      \put(9,38){\vector(-1,-2){8}}
      \put(11,38){\vector(1,-2){8}}
      \put(12,40){\vector(1,0){16}}
      \put(29,38){\vector(-1,-2){8}}
      \put(31,38){\vector(1,-2){8}}
      \put(2,20){\vector(1,0){16}}
      \put(22,20){\vector(1,0){16}}
      \put(19,18){\vector(-1,-2){8}}
      \put(21,18){\vector(1,-2){8}}
      \put(9,2){\vector(-1,2){8}}
      \put(12,0){\vector(1,0){16}}
      \qbezier(39,18)(35,10)(40,10)
      \qbezier(40,10)(45,10)(41,18)
      \put(41,18){\vector(-1,2){0}}
    \end{picture}
  \end{minipage}
\end{quote}
From rhe dependency graph and to a lesser extent from the original
equations, it is clear that ``E'' depends upon itself.  There is also
a cyclic dependency between ``C,'' ``D'' and ``F.''  Thus
topologically sorting and then evaluating them in this order will not
suffice to provide a solution.  No solution may exist.  But if the
values of the variables involved in cycles (\(\aset{\textrm{C},
  \textrm{D}, \textrm{E}, \textrm{F}}\)) are drawn from domains ordered
in ``partial 
orders'' and the
function ``p'' is ``monotone'' in its second argument, `q'' is monotone in
its third argument, ``r'' in its third argument and ``s'' in its
(only) argument, and if the  respective domains have distinguish
``bottom'' values and satisfy the
``ascending chain'' condition, then the values can be computed
according to the following schedule:
\begin{quote}
  \def\{{\char123}
  \def\}{\char125}
\begin{alltt}
A := m();
B := n(A);
C' := \(\bot\); D' := \(\bot\); F' := \(\bot\);
do \{ C := C'; D := D'; F := F';
     C' := p(A,F);
     D' := q(A,B,C);
     F' := s(D);
\} while (C \(\neq\) C' \(\vee\) D \(\neq\) D' \(\vee\) F \(\neq\) F');
E' := \(\bot\);
do \{ E := E';
     E' := r(E);
\} while (E \(\neq\) E')
G := t(D,F);
\end{alltt}
\end{quote}
In the evaluation, whenever we have a variable involve in a cyclic
depenency, we evaluate repeatedly from the ``bottom'' value until a
fixed point is reached.  For mutually dependent cyclic dependencies,
all variables are evaluated together in a group (as with ``C,'' ``D''
and ``F'').  Monotonicity ensures that the evaluation will not
oscillate between vaues, and the ascending chain condition ensurs that
the process will eventually terminate.

The variables not involved in cyclic dependencies (e.g., ``A,'' ``B''
and ``G'') are \emph{not} evaluated in a loop, and indeed it would
waste evauation time to do so.

Consider the following dependency graph with vertices for A, B, C, D, E, F, G
on the left of the following figure. On the right is a representation the quasi partial order for the dependency relation on the left.  Below, we have one possible quasi total order (schedule) for the dependency relation:

\begin{quote}
\begin{picture}(140,64)(0,-20)
  \put(10,40){\makebox(0,0){A}}
  \put(30,40){\makebox(0,0){B}}
  \put(0,20){\makebox(0,0){C}}
  \put(20,20){\makebox(0,0){D}}
  \put(40,20){\makebox(0,0){E}}
  \put(10,0){\makebox(0,0){F}}
  \put(30,0){\makebox(0,0){G}}
  \put(9,38){\vector(-1,-2){8}}
  \put(11,38){\vector(1,-2){8}}
  \put(12,40){\vector(1,0){16}}
  \put(29,38){\vector(-1,-2){8}}
  \put(31,38){\vector(1,-2){8}}
  \put(2,20){\vector(1,0){16}}
  \put(22,20){\vector(1,0){16}}
  \put(19,18){\vector(-1,-2){8}}
  \put(21,18){\vector(1,-2){8}}
  \put(9,2){\vector(-1,2){8}}
  \put(12,0){\vector(1,0){16}}
  \qbezier(39,18)(35,10)(40,10)
  \qbezier(40,10)(45,10)(41,18)
  \put(41,18){\vector(-1,2){0}}
  %
  \put(80,20){\makebox(0,0){A}}
  \put(100,20){\makebox(0,0){B}}
  \put(120,20){\makebox(0,0){\{C,D,F\}}}
  \put(140,40){\makebox(0,0){\{E\}}}
  \put(140,0){\makebox(0,0){G}}
  \put(82,20){\vector(1,0){16}}
  \put(102,20){\vector(1,0){10}}
  \put(124,24){\vector(1,1){13}}
  \put(124,16){\vector(1,-1){14}}
  %
  \put(40,-20){\makebox(0,0){A}}
  \put(60,-20){\makebox(0,0){B}}
  \put(80,-20){\makebox(0,0){\{C,D,F\}}}
  \put(100,-20){\makebox(0,0){\{E\}}}
  \put(120,-20){\makebox(0,0){G}}
  \put(42,-20){\vector(1,0){16}}
  \put(62,-20){\vector(1,0){10}}
  \put(88,-20){\vector(1,0){8}}
  \put(104,-20){\vector(1,0){14}}
\end{picture}
\end{quote}

\section{Formalities}

\subsection{Premilinaries}

First, let's recall some basic definitions:
\begin{itemize}
\item A relation $R$ over a set $S$ \(\left(R=(\sqsubseteq,S)\right)\) is
  \begin{itemize}
  \item \emph{reflexive} if \(\forall_{x\in S}\:  x \sqsubseteq x\)
  \item \emph{irreflexive} if \(\forall_{x\in S}\:  x \not\sqsubseteq x\)
  \item \emph{symmetric} if \(\forall_{x,y \in S}\: x \sqsubseteq y \implies y \sqsubseteq x \).
  \item \emph{anti-symmetric} if \(\forall_{x,y \in S}\: x \sqsubseteq y \wedge y \sqsubseteq x \implies x = y\).
    \item \emph{total} \(\forall_{x,y \in S}\: x \sqsubseteq y \vee y \sqsubseteq x\).
  \item \emph{transitive} if \( \forall_{x,y,z \in S}\: x \sqsubseteq y \wedge y \sqsubseteq z \implies x \sqsubseteq z\).
  \end{itemize}
\item A relation $R$ is a (strict) \emph{partial order} if it is both irreflexive and transitive.
  Note, as a consequence, a partial order is anti-symmetric, indeed has no symmeries at all.
  \item A relation $R$ is a \emph{preorder} if it is both reflexive and transitive.
  \item A relation $R$ is a \emph{total order} if it is a partial order and additionally total.  Analogously, a \emph{total preorder} is a preorder that is total.
\end{itemize}

We will use a variety of infix operators (e.g. \( <, \leq, \sim, \lesssim \)) as relations. 

\subsection{Definitions}

\def\SimHat#1{\stackrel{\sim}{#1}}
\def\NotSimHat#1{\stackrel{\not\sim}{#1}}

\begin{definition}
  A \emph{quasi (partial) order} is any relation $Q=(S, \lesssim)$ that is transitive.
  A \emph{quasi total order} is a quasi order that is total.
  Respective to the quasi order $Q$, we write \( x \sim y \) if and only if \( x \lesssim x \wedge y \lesssim x \), we write \( x \lnsim y \) if and only if \( x \lesssim y \wedge y \not\lesssim x \).  We define:
  \[ \SimHat{S} = \aset{x \mid x \sim x} \]
  \[ \NotSimHat{S} = \aset{x \mid x \not\sim x } \]
\end{definition}
\begin{consequence}
  For a quasi order $Q=(S, \lesssim)$, \((\SimHat{S},\sim)\) is an equivalence relation and 
  \((\NotSimHat{S},\lesssim)\) is a (strict) partial order.
\end{consequence}

\end{document}

% LocalWords:  maketitle emph qbezier
