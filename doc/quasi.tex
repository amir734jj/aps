\documentclass[12pt]{article}
  
\usepackage{fullpage}
\usepackage{pict2e}

\usepackage{alltt}
\usepackage{amssymb}
\usepackage{amsthm}

\def\implies{\Rightarrow}

\theoremstyle{definition}
\newtheorem{definition}{Definition}[section]

\theoremstyle{theorem}
\newtheorem{consequence}{Consequence}[section]
\newtheorem{theorem}{Theorem}[section]

\def\aset#1{\left\{{#1}\right\}}

\unitlength 1mm

\begin{document}

\title{Quasi Ordering for Dependency Scheduling in the Presence of Cycles}
\author{John Boyland and Amir Hesamian}
\date{DRAFT \today}

\maketitle

\begin{abstract}
  Unlike many graphs, a dependency graph may have loops (edges from a
  vertex to itself).  The presence of a loop makes a difference to the
  graph, since it represents a self-dependency.  Thus representing
  (the transitive closure of) a dependency graph as a partial order
  over strongly connected components loses some information: are the
  single-element components involved in a self-dependency or not?
  This information is needed for an evaluation (scheduling)
  mechanism.  We propose a new algebraic structure: the \emph{quasi
  order}, for which we propose partial and total versions.  Any
  transitive dependency relation can be represented as a quasi partial
  order, and any schedule of such a relation can used a quasi total order. 
\end{abstract}

\section{Motivation}

Consider the following set of equations that should be mutually
satisfied, and its dependency graph shown on the right:
\begin{quote}
  \begin{minipage}{2.75in}
\begin{verbatim}
A = m()
B = n(A)
C = p(A,F)
D = q(A,B,C)
E = r(B,D,E)
F = s(D)
G = t(D,F)
\end{verbatim}
  \end{minipage}\hfill
  \begin{minipage}{2.75in}
    \begin{picture}(40,44)(0,-4)
      \put(10,40){\makebox(0,0){A}}
      \put(30,40){\makebox(0,0){B}}
      \put(0,20){\makebox(0,0){C}}
      \put(20,20){\makebox(0,0){D}}
      \put(40,20){\makebox(0,0){E}}
      \put(10,0){\makebox(0,0){F}}
      \put(30,0){\makebox(0,0){G}}
      \put(9,38){\vector(-1,-2){8}}
      \put(11,38){\vector(1,-2){8}}
      \put(12,40){\vector(1,0){16}}
      \put(29,38){\vector(-1,-2){8}}
      \put(31,38){\vector(1,-2){8}}
      \put(2,20){\vector(1,0){16}}
      \put(22,20){\vector(1,0){16}}
      \put(19,18){\vector(-1,-2){8}}
      \put(21,18){\vector(1,-2){8}}
      \put(9,2){\vector(-1,2){8}}
      \put(12,0){\vector(1,0){16}}
      \qbezier(39,18)(35,10)(40,10)
      \qbezier(40,10)(45,10)(41,18)
      \put(41,18){\vector(-1,2){0}}
    \end{picture}
  \end{minipage}
\end{quote}
From the dependency graph and to a lesser extent from the original
equations, it is clear that ``E'' depends upon itself.  There is also
a cyclic dependency between ``C,'' ``D'' and ``F.''  Thus
topologically sorting and then evaluating them in this order will not
suffice to provide a solution.  No solution may exist.  But if the
values of the variables involved in cycles (\(\aset{\textrm{C},
  \textrm{D}, \textrm{E}, \textrm{F}}\)) are drawn from domains ordered
in ``partial 
orders'' and the
function ``p'' is ``monotone'' in its second argument, `q'' is monotone in
its third argument, ``r'' in its third argument and ``s'' in its
(only) argument, and if the  respective domains have distinguish
``bottom'' values and satisfy the
``ascending chain'' condition, then the values can be computed
according to the following schedule:
\begin{quote}
  \def\{{\char123}
  \def\}{\char125}
\begin{alltt}
A := m();
B := n(A);
C' := \(\bot\); D' := \(\bot\); F' := \(\bot\);
do \{ C := C'; D := D'; F := F';
     C' := p(A,F);
     D' := q(A,B,C);
     F' := s(D);
\} while (C \(\neq\) C' \(\vee\) D \(\neq\) D' \(\vee\) F \(\neq\) F');
E' := \(\bot\);
do \{ E := E';
     E' := r(E);
\} while (E \(\neq\) E')
G := t(D,F);
\end{alltt}
\end{quote}
In the evaluation, whenever we have a variable involve in a cyclic
dependency, we evaluate repeatedly from the ``bottom'' value until a
fixed point is reached.  For mutually dependent cyclic dependencies,
all variables are evaluated together in a group (as with ``C,'' ``D''
and ``F'').  Monotonicity ensures that the evaluation will not
oscillate between values, and the ascending chain condition ensures that
the process will eventually terminate.

The variables not involved in cyclic dependencies (e.g., ``A,'' ``B''
and ``G'') are \emph{not} evaluated in a loop, and indeed it would
waste evaluation time to do so.

Consider the following dependency graph with vertices for A, B, C, D, E, F, G
on the left of the following figure. On the right is a representation the quasi partial order for the dependency relation on the left.  Below, we have one possible quasi total order (schedule) for the dependency relation:

\begin{quote}
\begin{picture}(140,64)(0,-20)
  \put(10,40){\makebox(0,0){A}}
  \put(30,40){\makebox(0,0){B}}
  \put(0,20){\makebox(0,0){C}}
  \put(20,20){\makebox(0,0){D}}
  \put(40,20){\makebox(0,0){E}}
  \put(10,0){\makebox(0,0){F}}
  \put(30,0){\makebox(0,0){G}}
  \put(9,38){\vector(-1,-2){8}}
  \put(11,38){\vector(1,-2){8}}
  \put(12,40){\vector(1,0){16}}
  \put(29,38){\vector(-1,-2){8}}
  \put(31,38){\vector(1,-2){8}}
  \put(2,20){\vector(1,0){16}}
  \put(22,20){\vector(1,0){16}}
  \put(19,18){\vector(-1,-2){8}}
  \put(21,18){\vector(1,-2){8}}
  \put(9,2){\vector(-1,2){8}}
  \put(12,0){\vector(1,0){16}}
  \qbezier(39,18)(35,10)(40,10)
  \qbezier(40,10)(45,10)(41,18)
  \put(41,18){\vector(-1,2){0}}
  %
  \put(80,20){\makebox(0,0){A}}
  \put(100,20){\makebox(0,0){B}}
  \put(120,20){\makebox(0,0){\{C,D,F\}}}
  \put(140,40){\makebox(0,0){\{E\}}}
  \put(140,0){\makebox(0,0){G}}
  \put(82,20){\vector(1,0){16}}
  \put(102,20){\vector(1,0){10}}
  \put(124,24){\vector(1,1){13}}
  \put(124,16){\vector(1,-1){14}}
  %
  \put(40,-20){\makebox(0,0){A}}
  \put(60,-20){\makebox(0,0){B}}
  \put(80,-20){\makebox(0,0){\{C,D,F\}}}
  \put(100,-20){\makebox(0,0){\{E\}}}
  \put(120,-20){\makebox(0,0){G}}
  \put(42,-20){\vector(1,0){16}}
  \put(62,-20){\vector(1,0){10}}
  \put(88,-20){\vector(1,0){8}}
  \put(104,-20){\vector(1,0){14}}
\end{picture}
\end{quote}

\section{Formalities}

\subsection{Preliminaries}

First, let's recall some basic definitions:
\begin{itemize}
\item A (binary) relation $R$ over a set $S$ \(\left(R=(\sqsubseteq,S)\right)\) is
  \begin{itemize}
  \item \emph{reflexive} if \(\forall_{x\in S}\:  x \sqsubseteq x\)
  \item \emph{irreflexive} if \(\forall_{x\in S}\:  x \not\sqsubseteq x\)
  \item \emph{symmetric} if \(\forall_{x,y \in S}\: x \sqsubseteq y \implies y \sqsubseteq x \).
  \item \emph{anti-symmetric} if \(\forall_{x,y \in S}\: x \sqsubseteq y \wedge y \sqsubseteq x \implies x = y\).
    \item \emph{total} \(\forall_{x \neq y \in S}\: x \sqsubseteq y \vee y \sqsubseteq x\).
  \item \emph{transitive} if \( \forall_{x,y,z \in S}\: x \sqsubseteq y \wedge y \sqsubseteq z \implies x \sqsubseteq z\).
  \end{itemize}
\item A relation $R$ is a (strict) \emph{partial order} if it is both irreflexive and transitive.
  Note, as a consequence, a partial order is anti-symmetric, indeed has no symmetries at all.
  \item A relation $R$ is a \emph{preorder} if it is both reflexive and transitive.
  \item A relation $R$ is a \emph{total order} if it is a partial order and additionally total.  Analogously, a \emph{total preorder} is a preorder that is total.
  \item We define $R_1 \subseteq R_2$ for \( R_i = (\sqsubseteq_i,S_i)\) as
    \[
    \forall_{a,b \in S_1} a \sqsubseteq_1 b \implies \left( a,b \in S_2 \wedge
    a \sqsubseteq_2 b \right)
    \]
  \item We define the union of a family of relations: \( R = \bigcup_i R_i
    \) where \( R_i = (\sqsubseteq_i, S_i) \) as \( (\sqsubseteq,S) \)
    where
    \begin{eqnarray*}
      S &=& \bigcup_i S_i \\
      a \sqsubseteq b &\textrm{iff}& \exists_i\: a,b \in S_i \wedge a
      \sqsubseteq_i b
    \end{eqnarray*}
    \item We define the restriction of a relation
      \(R=(\sqsubseteq,S)\) to a subset $S'\subseteq S'$ (written
      $R\mid_{S'}$) as \(R' =
      (\sqsubseteq',S')\) where
      \[
      a \sqsubseteq' b \textrm{ if and only if } a,b \in S' \wedge a
      \sqsubseteq b
      \]
\end{itemize}

\begin{definition}
    A \emph{transitive closure} of a relation $R$ over a set $S$ is a
    smallest transitive relation $R'$ over the same set $S$ such that
    $R \subseteq R'$.
\end{definition}
\begin{theorem}
  A transitive closure is always defined and is unique.  Therefore, we
  speak of \emph{the} transitive closure of a relation.
\end{theorem}

We will use a variety of infix operators (e.g. \( <, \leq, \sim, \lesssim \)) as relations. 

\subsection{Definitions}

\def\SimHat#1{\stackrel{\sim}{#1}}
\def\NotSimHat#1{\stackrel{\not\sim}{#1}}

\begin{definition}
  A \emph{quasi (partial) order} is any relation $Q=(\lesssim,S)$ that is transitive.
  A \emph{quasi total order} is a quasi order that is total.
  Respective to the quasi order $Q$, we write \( x \sim y \) if and only if \( x \lesssim x \wedge y \lesssim x \), we write \( x \lnsim y \) if and only if \( x \lesssim y \wedge y \not\lesssim x \).  We define:
  \[ \SimHat{S} = \aset{x \mid x \sim x} \]
  \[ \NotSimHat{S} = \aset{x \mid x \not\sim x } \]
\end{definition}
\begin{theorem}
  For a quasi order $Q=(\lesssim,S)$, \((\sim,\SimHat{S})\) is an equivalence relation and 
  \((\lesssim,\NotSimHat{S})\) is a (strict) partial order, and total
  if $Q$ is total.
\end{theorem}
\begin{proof}
  
  \verb| |
  
  \begin{enumerate}
    \item
      From its definition, $\sim$ is symmetric.  It is transitive since if
      \( x \sim y \wedge y \sim z \), we have \( x \lesssim y \wedge y
      \lesssim x \wedge y \lesssim z \wedge z \lesssim y \), from which
      by transitivity of $\lesssim$, we have \( x \lesssim z \wedge z
      \lesssim x \) from which \( x \sim z \).  Finally, $\sim$ is
      reflexive over \(\SimHat{S}\) by definition.
    \item
      By definition $\lesssim$ is transitive and it is irreflexive over
      $\NotSimHat{S}$ by definition of that set.  Thus it is a partial
      order over $\NotSimHat{S}$.  Furthermore, if $Q$ is total, then
      so is \((\lesssim,\NotSimHat{S})\).
  \end{enumerate}
\end{proof}

Total orders are incompatible: they serve as the bottom (incomparable)
elements in a (semi-)lattice of partial orders.  But every partial order is a
quasi order, and when we form the lattice of quasi orders, we can
combine total orders into a quasi-order.  The lattice has a proper
bottom element: the ``complete'' binary relation which is a quasi
total order.  The following result demonstrates the ability of quasi
total orders to encompass multiple total orders:

\begin{theorem}
  The transitive closure of a non-empty union of total orders over the
  same set $S$ is a
  quasi total order over $S$.
\end{theorem}
\begin{proof}
  Let \( Q \) be the transitive closure of \(R = \bigcup_i R_i\).  Clearly
  $Q$ is transitive (by definition of transitive closure).  And it is
  total since all two elements of $S$ are related in every one of the
  (non-zero) total orders.  Thus it is a quasi total order.
\end{proof}

\subsection{The Cycle-Free Representation}

It is convenient to distill a quasi order down to a partial order
using the equivalence classes.  This construction is similar to the
strongly-connected component representation of a preorder, but
distinguishes elements that have self-edges from those that do not.

\begin{definition}
  The \emph{cycle-free} representation of a quasi-order
  $Q=(\lesssim,S)$ (written $[Q]$) is the relation \( [Q] = \left(<,
  \NotSimHat{S} \cup \SimHat{S}\!\!/{\mathord\sim}\right) \) over the irreflexive
  subset of the base set together with the quotient set of the
  reflexive set where
  \begin{eqnarray*}
    x < y &\textrm{iff}& x \lnsim y \\
    {[\aset{u,\ldots}]} < y &\textrm{iff}& u \lnsim y \\
    x < [\aset{v,\ldots}] & \textrm{iff}& x \lnsim v \\
    {[\aset{u,\ldots}]} < [\aset{v,\ldots}] & \textrm{iff} & u \lnsim
    v
  \end{eqnarray*}
\end{definition}
\begin{theorem}
  The cycle-free representation of $Q$ is well-defined, unique and is a
  partial order, and additionally is total if and only if $Q$ is total.
\end{theorem}
\begin{proof}
  We prove the four claims sequentially:
  \begin{itemize}
  \item $[Q]$ is well defined.\par
    To be well defined, we need the representation element used to
    define $<$ to be unimportant: all members of the set yield the
    same result.  In particular, we need that for any element $x \in
    S$ and any equivalence class
    $E \in {\SimHat{S}\!\!/{\mathord\sim}}$, then $\exists v\in E: x \lnsim v$
    then $\forall v \in E: x \lnsim v$ and also the reverse property
    $\exists u\in E: u \lnsim x$ implies
    $\forall u\in E: u \lnsim x$.

    To prove the first, let $x \in S$ and $v \in E$ where
    $x \lnsim v$.  Suppose we have an arbitrary $v' \sim v$, then
    $v \lesssim v'$ and thus by transitivity, $x \lesssim v'$.
    Suppose, contrary-wise, $v' \lesssim x$, then by equivalence and
    transitivity, we have $v \lesssim x$ which contradicts our
    assumption.  Therefore $x \lnsim v'$.

    The reverse property is proved analogously.

  \item $[Q]$ is unique.  In other words,
    if $[Q] = [Q']$ then $Q = Q'$.

    Suppose $[Q] = [Q']$ for quasi orders $Q$ and $Q'$,
    and further suppose $x \lesssim y$.  We will prove that
    $x \lesssim' y$ and thus since no generality was lost, we have $Q = Q'$.

    If $x \sim y$, then $x,y \in E$ where $E$ is an equivalence class.
    This equivalence class is an element of $[Q]$ and thus must also be
    in $[Q']$ and thus $x \sim' y$ and the result is proved.

    If on the other hand, we have $x \lnsim y$, then consider whether
    each is in $\SimHat{S}$ (for $Q$).  Suppose $x \in E_1$,
    $y \in E_2$ equivalence classes of $\sim$.  Then these equivalence
    classes are in $[Q]$ with $E_1 < E_2$, and thus in $[Q']$ (with
    $E_1 <' E_2$) and we have (by our previous result) that
    $x \lnsim' y$.   Then if $x \in \NotSimHat{S}$ and $y \in E$, we
    have $x < E$ and thus $x <' E$ and thus $x \lnsim' y$.  The other
    cases are analogous.

  \item $[Q]$ is a partial order.

    $[Q]$ must be irreflexive.  Otherwise we would have $x \lnsim x$
    for some $x \in S$ which is a contradiction.
    
    Furthermore $[Q]$ is transitive, since $\lnsim$ is transitive.

  \item $[Q]$ is total if and only if $Q$ is total.
    \begin{description}
    \item[if]  Suppose $Q$ is total.  Then consider two elements of
      the base set of $[Q]$.

      If we have two equivalence classes, then if they are the same,
      we have nothing to prove for totality.  If they are different,
      then by the totality of $Q$, for some representatives $x$ and
      $y$ of the respective equivalence classes,
      we must have $x \lesssim y$ or $y \lesssim x$.  We cannot have
      both because otherwise we would have $x \sim y$ and they would
      be in the same equivalence class.  Thus we have $x \lnsim y$ or
      $y \lnsim x$ which shows that the two equivalence classes are
      related in $[Q]$.

      If we have an element $x \in \NotSimHat{S}$ and an equivalence
      class $E$, then for a representative $y \in E$, we can make the
      same argument as above to show we have totality.

      Finally if we are considering \( x \neq y\) both in \(
      \NotSimHat{S} \), then
      we can again make the same argument.

    \item[only if]  Suppose $[Q]$ is total, then consider $x\neq y$.

      Suppose $x$ and $y$ are in equivalence classes $E_1$ and $E_2$
      respectively.  If $E_1 = E_2$, then $x \lesssim y$
      (and $y \lesssim x$).  Otherwise, since $[Q]$ is total, we must
      have either $E_1 < E_2$ or $E_2 < E_1$.  In the first case, this
      means $x \lnsim y$, and in the second case $y \lnsim x$.
      In either case, $x$
      and $y$ are related.

      Suppose $x$ is in an equivalence class $E$ and $y$ is not.
      Then by the totality of $[Q]$, either $E < y$ or
      $y < E$.  By the definition of $<$, this means either $x \lnsim
      y$ or \( y \lnsim x\).
      
      The case of $x$ not being in an equivalence class while $y$
      is in an equivalence class $E$ is analogously proved.

      Finally suppose that neither $x$ nor $y$ is in an equivalence
      class.  Then by totality of $[Q]$, they must be related by $Q$.
    \end{description}
  \end{itemize}
\end{proof}

Indeed the construction can be carried out in the other direction too.
To do so, we define an ungainly term for the result:
\begin{definition}
  For a set $S$, a \emph{semi partition partial order} is a partial
  order over a set \(S^{\sharp} = S_0 \cup \aset{S_1,\ldots}\), where  \(S =
  S_0 \cup S_1 \cup \ldots \) ($S_i\neq \emptyset$ for all
  positive $i$) is a partition of $S$ where only the first subset can
  be empty. The elements of
  \(S^{\sharp}\) are the elements of $S_0$ (which may be empty) plus
  the remaining sets of the partition (which may have no further sets).
\end{definition}
\begin{theorem}
  For any set S, there is a bijection from the set of quasi-orders of
  S to the set of semi partition partial orders.
\end{theorem}
\begin{proof}
  For a partial order $Q=(\lesssim,S)$, $[Q]$ is a semi partition
  partial order for $S$ where the partition is $\NotSimHat{S}$ plus
  the equivalence classes of $\SimHat{S}$. And we have shown that the
  construction is unique (one-to-one).  It remains to show that is
  ``onto'' (surjective).

  Let $P=(<,S^{\sharp})$ be a semi partition partial order for
  $S = S_0 \cup S_1 \cup \ldots$ .  We
  construct a quasi order $Q=(\lesssim,S)$ such that $P = [Q]$.
  The relation $\lesssim$ is defined as follows:
  \[
  x \lesssim y \textrm{ iff }
  \left\{
  \begin{array}{cl}
    \textrm{true}& \textrm{if } x,y \in S_i, i > 0 \\
    S_i < S_j & \textrm{if } x \in S_i, i > 0, y \in S_j, i \neq j > 0 \\
    x < S_j   & \textrm{if } x \in S_0, y \in S_j, j > 0 \\     
    S_i < y   & \textrm{if } x \in S_i, i > 0, y \in S_0 \\
    x < y     & \textrm{if } x, y \in S_0
  \end{array}\right.
  \]
  The transitivity of $Q$ follows from the transitivity of $P$, and
  clearly $[Q] = P$.
\end{proof}

As a consequence of this construction, we can linearize a quasi order
to a quasi total order without disturbing the equivalence classes: we
convert to the cycle-free representation, topologically sort that and
then convert back to a quasi (now total) order.

We can actually do this in a topological sort of the original
dependency relation (which $Q$ would be the transitive closure of).

\end{document}

% LocalWords:  maketitle emph qbezier Boyland Amir Hesamian preorder
% LocalWords:  Monotonicity irreflexive iff bijection surjective
% LocalWords:  linearize
